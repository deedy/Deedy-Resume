%%%%%%%%%%%%%%%%%%%%%%%%%%%%%%%%%%%%%%%
% Deedy CV/Resume
% XeLaTeX Template
% Version 1.0 (5/5/2014)
%
% This template has been downloaded from:
% http://www.LaTeXTemplates.com
%
% Original author:
% Debarghya Das (http://www.debarghyadas.com)
% With extensive modifications by:
% Vel (vel@latextemplates.com)
%
% License:
% CC BY-NC-SA 3.0 (http://creativecommons.org/licenses/by-nc-sa/3.0/)
%
% Important notes:
% This template needs to be compiled with XeLaTeX.
%
%%%%%%%%%%%%%%%%%%%%%%%%%%%%%%%%%%%%%%

\documentclass[letterpaper]{deedy-resume} % Use US Letter paper, change to a4paper for A4
%\documentclass[a4paper]{deedy-resume} % Use US Letter paper, change to a4paper for A4

\usepackage{xeCJK}

\begin{document}

%----------------------------------------------------------------------------------------
%	TITLE SECTION
%----------------------------------------------------------------------------------------

%\lastupdated % Print the Last Updated text at the top right
\jobapplied{控制工程师}

\namesection{康}{昌霖}{ % Your name
\urlstyle{same}\url{http://cn.linkedin.com/in/kangchanglin} \\ % Your website, LinkedIn profile or other web address
\href{mailto:kangchanglin0509@hotmail.com}{kangchanglin0509@hotmail.com} | TEL:XXX-XXXX-XXXX % Your contact information
}

%----------------------------------------------------------------------------------------
%	LEFT COLUMN
%----------------------------------------------------------------------------------------

\begin{minipage}[t]{0.30\textwidth} % The left column takes up 33% of the text width of the page

%------------------------------------------------
% Education
%------------------------------------------------

\section{教育背景}

\subsection{浙江大学}
\descript{工学硕士 \textbullet{} 控制科学与工程}
\location{就读时间:2014.9 - 2017.3 \\ 研究方向:控制科学与控制工程 \\  GPA: 89.89/100}

\sectionspace % Some whitespace after the section

\subsection{厦门大学}

\descript{工学学士 \\ 机械设计制造及其自动化专业}
\location{就读时间:2010.9 - 2014.6 \\  专业排名: 5/96}

\sectionspace % Some whitespace after the section

\descript{经济学学士 \textbullet{} 经济学(辅修)}
\location{就读时间:2011.9 - 2014.6
%\\  GPA: ??/100
}

\sectionspace % Some whitespace after the section

%------------------------------------------------
% Coursework
%------------------------------------------------

\section{课程}

\subsection{硕士}

线性系统理论 ,最优化与最优控制\\
矩阵论 ,系统辨识与滤波\\
模式识别与人工智能

\sectionspace % Some whitespace after the section

%------------------------------------------------

\subsection{本科}

理论力学,材料力学,流体力学 \\
微机原理与接口技术,机电一体化 \\
自动控制原理,计算机图形学 \\

\sectionspace % Some whitespace after the section

%------------------------------------------------
% Skills
%------------------------------------------------

\section{技能}
\subsection{编程}
\descript{熟练(超过5000行):}
\location{Matlab \textbullet{} \LaTeX\
}
\descript{使用过(超过1000行):}
\location{C(单片机代码编写) \textbullet{}VC++(WinAPI和QT)\textbullet{} Android \textbullet{} 汇编
}
\descript{了解:}
\location{Labview \textbullet{} ARM编程}

\sectionspace % Some whitespace after the section

\subsection{设计}
\descript{机械设计:}
\location{三维建模 \textbullet{} AutoCAD \textbullet{} ANSYS
}
\descript{电子设计:}
\location{Altium Designer \textbullet{} Multisim
}
\descript{其它:}
\location{幻灯片排版设计 \textbullet{} MS Visio}

\sectionspace % Some whitespace after the section
\subsection{英语}
\location{TOEFL iBT: 90(2013.8)}
%------------------------------------------------
% Links
%------------------------------------------------

%\section{链接}
%
%LinkedIn:// \href{https://cn.linkedin.com/in/kangchanglin}{\bf KANG Changlin}


%\sectionspace % Some whitespace after the section


%----------------------------------------------------------------------------------------

\end{minipage} % The end of the left column
\hfill\hfill\hfill
%
%----------------------------------------------------------------------------------------
%	RIGHT COLUMN
%----------------------------------------------------------------------------------------
%
\begin{minipage}[t]{0.65\textwidth} % The right column takes up 66% of the text width of the page

%------------------------------------------------
% Experience
%------------------------------------------------

%\section{EXPERIENCE}
%
%\runsubsection{Coursera}
%\descript{| KPCB Fellow + Software Engineering Intern}
%
%\location{Expected June 2014 – Sep 2014 | Mountain View, CA}
%\vspace{\topsep} % Hacky fix for awkward extra vertical space
%\begin{tightitemize}
%\item 52 out of 2500 applicants chosen to be a KPCB Fellow 2014.
%\end{tightitemize}
%
%\sectionspace % Some whitespace after the section
%
%%------------------------------------------------
%
%\runsubsection{Google}
%\descript{| Software Engineering Intern}
%
%\location{May 2013 – Aug 2013 | Mountain View, CA}
%\begin{tightitemize}
%\item Worked on the YouTube Captions team in primarily vanilla Javascript and Python to plan, design and develop the full stack implementation of a new framework to add and edit Automatic Speech Recognition captions.
%\item Created a backbone.js-like framework for the Captions editor.
%\item All code was reviewed, perfected, and pushed to production.
%\end{tightitemize}
%
%\sectionspace % Some whitespace after the section
%
%%------------------------------------------------
%
%\runsubsection{Phabricator}
%\descript{| Open Source Contributor \& Team Leader}
%
%\location{Jan 2013 – May 2013 | Palo Alto, CA \& Ithaca, NY}
%\begin{tightitemize}
%\item Phabricator is used daily by Facebook, Dropbox, Quora, Asana and more.
%\item I created the Meme generator, the entire Lipsum application, ported Tokens to different apps, fixed many bugs and more in PHP and Shell.
%\item Led a team from MIT, Cornell, IC London and UHelsinki for the project.
%\end{tightitemize}
%
%\sectionspace % Some whitespace after the section

%------------------------------------------------
% Research
%------------------------------------------------

\section{项目经历}

\runsubsection{作业型水下自主机器人系统开发及其相关控制研究}
\descript{| 研究生负责人}

\location{2014.9 – 现在 | 浙江大学}
负责协助导师系统设计,项目执行管理和任务分配。并完成相关具体的设计研究工作。团队包括一名研究生和一名本科生,曾协助导师指导一份本科毕设。\textbf{两项发明专利已公开,一篇会议论文即将发表}。具体地:\\
\vspace{\topsep}
\begin{tightitemize}
\item 水下机器人浮沉控制装置的设计开发和控制研究:已公开两项发明专利;已提交一份会议论文,即将发表;协助指导一位本科毕业生毕设。系统动力学建模,时间最优控制算法的证明推导,并使用Matlab仿真验证;分别基于Matlab和Android开发了上位机控制程序;下位机控制电路设计;机械设计(包括液压系统、防水设计等)、交外加工和组装调试。
\item 水下机器人缩小验证模型的制作:上位机(基于QT\&WinAPI)控制端实现,包括对3D鼠标(3DConnexion)的二次开发、串口操作、界面操作、多线程等工作内容。姿态稳定控制算法设计、仿真和实现。机械设计。
\item 水下机器人机体系统设计,控制、电气架构设计。机架结构设计。相关控制算法设计、仿真。
\end{tightitemize}
\sectionspace % Some whitespace after the section

%%------------------------------------------------

\runsubsection{非球面轴承的研制(挑战杯)}
\descript{| 队员}

\location{2012.11 – 2013.4 | 厦门大学}
主要负责轴承实体的加工和实验,专利撰写,参赛文本中的插图制作及原理部分的撰写。团队作品取得\textbf{校级特等奖}。

\sectionspace % Some whitespace after the section

%------------------------------------------------

\runsubsection{分形演示软件(计算机图形学课程项目)}
\descript{| 独自完成}

\location{2012.12 | 厦门大学}
在VC环境下制作了一个小程序,用于实现利用分形绘制Caley树的功能,其中树干长度、绘制起点固定,树干和树枝夹角、迭代次数和树干树枝比例可变,由输入值指定。

\sectionspace % Some whitespace after the section

%------------------------------------------------
% Awards
%------------------------------------------------

\section{学术成果}
\subsection{专利}
\begin{tabular}{rll}
2015	 & 发明专利 &                       水下机器人自排油浮力调节装置, CN105173040A\\
2015	 & 发明专利 &                       活塞式浮力调节装置, CN104828222A\\
2014	 & 实用新型 &                       一种辊压精成形机齿条高度调节装置, CN201420071039\\
2013	 & 实用新型 &                       贴片式无线传输电子运动采集装置, CN201320668356\\
2013     & 实用新型 &                       一种用于小功率电器的无线充电装置, CN203119584U\\
\end{tabular}

\sectionspace % Some whitespace after the section

\subsection{论文}
\begin{tabular}{rl}
2016    & "Development of a Hydraulic Buoyance Adjustment Module for  \\
        & Underwater Robots", 即将发表\\
2013    & "A Technology of CNC Punching of Chinese Characters" , 已发表\\
\end{tabular}

\sectionspace % Some whitespace after the section
%------------------------------------------------
% Societies
%------------------------------------------------

\section{获奖与荣誉}

\begin{tabular}{rll}
2013.12 &中国电机工程学会杯全国大学生数学建模竞赛& 全国二等奖    \\
2013.5  &第十三届”挑战杯“学生课外学术科技作品竞赛&           校级特等奖    \\
%2013.2  &美国大学生数学建模大赛&                             Successful    \\
2014.4  &厦门大学物理与机电工程学院萨本栋奖学金& \\
2012.9  &厦门大学校一级奖学金&       \\
%2011.9  &厦门大学校一级奖学金&       \\
2014.6  &厦门大学优秀三好学生&       \\
2014.6  &厦门大学优秀毕业设计&       \\
\end{tabular}

\sectionspace % Some whitespace after the section

%----------------------------------------------------------------------------------------

\end{minipage} % The end of the right column

%----------------------------------------------------------------------------------------
%	SECOND PAGE (EXAMPLE)
%----------------------------------------------------------------------------------------

%\newpage % Start a new page

%\begin{minipage}[t]{0.33\textwidth} % The left column takes up 33% of the text width of the page

%\section{Example Section}

%\end{minipage} % The end of the left column
%\hfill
%\begin{minipage}[t]{0.66\textwidth} % The right column takes up 66% of the text width of the page

%\section{Example Section 2}

%\end{minipage} % The end of the right column

%----------------------------------------------------------------------------------------

\end{document}
